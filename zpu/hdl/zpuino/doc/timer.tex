\section{Timers}
ZPUino includes two generic timers, which are able to generate PWM signals. 
The default implementation includes a 16-bit timer with prescaler and TSC, and a second 24-bit timer with no prescaler nor TSC.
The PWM controls of the 2nd timer are double buffered.

\subsection{HDL sources and modules}

\subsubsection{HDL instantiation template}
\begin{lstlisting}[language=VHDL]
component zpuino_timers is
  generic (
    A_TSCENABLED: boolean := false;
    A_PWMCOUNT: integer range 1 to 8 := 2;
    A_WIDTH: integer range 1 to 32 := 16;
    A_PRESCALER_ENABLED: boolean := true;
    A_BUFFERS: boolean :=true;
    B_TSCENABLED: boolean := false;
    B_PWMCOUNT: integer range 1 to 8 := 2;
    B_WIDTH: integer range 1 to 32 := 16;
    B_PRESCALER_ENABLED: boolean := false;
    B_BUFFERS: boolean := false
  );

  port (
    wb_clk_i: in std_logic;
    wb_rst_i: in std_logic;
    wb_dat_o: out std_logic_vector(wordSize-1 downto 0);
    wb_dat_i: in std_logic_vector(wordSize-1 downto 0);
    wb_adr_i: in std_logic_vector(maxIObit downto minIObit);
    wb_we_i:  in std_logic;
    wb_cyc_i: in std_logic;
    wb_stb_i: in std_logic;
    wb_ack_o: out std_logic;
    wb_inta_o:out std_logic;
    wb_intb_o:out std_logic;
    
    pwm_A_out: out std_logic_vector(A_PWMCOUNT-1 downto 0);
    pwm_B_out: out std_logic_vector(B_PWMCOUNT-1 downto 0)
  );
end component;
\end{lstlisting}

\subsubsection{Compliance}
The Timer module is wishbone compatible, in non-pipelined mode.

\subsubsection{Generics}

There are two sets of generics, one for each timer (timer A and B)

\begin{description}
\item{TSCENABLED} \hfill \\ Set if TSC (TimeStamp Counter) is enabled on timer A/B(default: false)
\item{PWMCOUNT} \hfill \\ Number of PWM outputs for timer A/B (default: 2)
\item{WIDTH} \hfill \\ Timer A/B width, in bits (default: 16)
\item{PRESCALER\_ENABLED} \hfill \\ Set if prescaler is enabled for timer A/B (default: true for timer A, false for timer B)
\item{BUFFERS} \hfill \\ Set if double-buffering of PWM registers is enabled on this timer (default: false for timer A, true for timer B)
\end{description}


\subsubsection{Source files}

ZPUino Timers are composed of the following source modules:

\begin{table}[h!]
\begin{center}
\begin{tabularx}{10cm}{ll}

zpuino\_timers.vhd & Top level Timers module \\
timer.vhd & Timer module \\
prescaler.vhd & Prescaler module
\end{tabularx}
\end{center}
\caption{Timer source files}
\end{table}

\subsection{Location}

Timer registers are usualy located in IOSLOT 3.

\subsection{Registers}

\subsubsection{TMRxCTL register}

TMRxCTL is located at offset $0$ for the first timer, and offset $64$ for the second timer.

\begin{table}[H]
\begin{center}
\begin{tabularx}{14cm}{Xcccccccc}
31 \hfill 11 & 10 \hfill 9 & 8 & 7 & 6  \hfill 4 & 3 & 2 & 1 & 0  \\
\hline
\multicolumn{1}{|c|}{\tiny rsvd} & 
\multicolumn{1}{|c|}{\tiny TMRUPDP} &
\multicolumn{1}{|c|}{\tiny rsvd\footnote{On pre-1.0 release, this used to be a OCREN register, but it's no longer used}} &
\multicolumn{1}{|c|}{\tiny TMRINTR} &
\multicolumn{1}{|c|}{\tiny TMRPRES} &
\multicolumn{1}{|c|}{\tiny TMRIEN} &
\multicolumn{1}{|c|}{\tiny TMRDIR} &
\multicolumn{1}{|c|}{\tiny TMRCCM} &
\multicolumn{1}{|c|}{\tiny TMREN}  \\
\hline
\end{tabularx}
\caption{TMRxCTL register}
\end{center}
\end{table}


\begin{description}

\item{0 - TMREN [RW]} \hfill \\
Timer Enable. When set to '1', timer will count either up or down, depending on $TMRDIR$ value.

\item{1 - TMRCCM [RW]} \hfill \\
Timer Clear on Compare Match. If this bit is set, whenever $TMRCNT$ matches $TMRCMP$ the current
value of $TMRCNT$ will be set to zero.

\item{2 - TMRDIR [RW]}\hfill \\
Timer count direction. If this bit is set, then at every timer tick the value of $TMRCNT$ will be incremented. If this
bit is not set, then $TMRCNT$ value will be decremented instead.

\item{3 - TMRIEN [RW]} \hfill \\
Timer Interrupt Enable. When this bit is set and $TMRINTR$ is not set, an interrupt is generated whenever 
$TMRCNT$ matches $TMRCMP$.
To rearm interrupt $TMRINTR$ must be cleared before exiting the ISR routine.

\item{[6-4] TMRPRES [RW]} \hfill \\
Timer prescaler. The timer prescale has 3 bits. See table \ref{timerprescaler} for more details on the prescaler value.
                 
\item{7 - TMRINTR [RW]} \hfill \\
Timer Interrupt. This bit is set whenever an interrupt occurs. It needs to be cleared in software to re-enable this interrupt
source.

\item{8 - Reserved} \hfill \\
Reserved.

\item{[10:9] - TMRUPDP [RW]} \hfill \\
Timer PWM update policy. See table \ref{updatepolicy} for details.

\end{description}


\subsubsection{TMRxCNT}
TMRxCNT is located at offset $1$ for the first timer, and offset $65$ for the second timer. \\
Current timer counter value. This is a read/write register.

\subsubsection{TMRxCMP}

TMRxCMP is located at offset $2$ for the first timer, and offset $66$ for the second timer. \\
Current timer compare value. This is a read-write register.

\subsubsection{TMRxTSC}

TMRxCMP is located at offset $3$ for the first timer, and offset $67$ for the second timer. \\
Timestamp Counter. This register is only available in the first timer (by omission). It's also known as $TIMERTSC$.
This register is incremented at each clock cycle, and is 32-bit wide. It's a read-only register.

\subsection{PWM registers}

The PWM output, is, unlike common PWM implementations, controlled by a LOW and a HIGH register. This allows for 
more precise operation. \\
The exact location of PWM registers are a bit complex, since each timer might have zero or more PWM outputs.
The address for PWM output $Y$ of timer $X$ can be computed by the following equation:

\begin{displaymath}
TMRxPWMBASE = ( 64 \times X) + 32 + ( 4 \times Y )
\end{displaymath}

Use the provided "C" macros whenever possible. See section \ref{timersoftware} for more details.

\subsubsection{TMRxPWMLOW}
TMPxPWMLOW is located at offset $0$ from TMRxPWMBASE.\\
Low PWM compare value. If $TMRxCNT$ is higher or equal this value, the PWM will output 0. It's a write-only register.
\subsubsection{TMRxPWMHIGH}
TMPxPWMLOW is located at offset $1$ from TMRxPWMBASE.\\
High PWM compare value. If $TMRxCNT$ is lower than this value, the PWM will output 1. It's a write-only register.
\subsubsection{TMRxPWMCTL}
TMPxPWMLOW is located at offset $2$ from TMRxPWMBASE.\\
PWM control bits. It's a write-only register.

\begin{table}[H]
\begin{center}
\begin{tabularx}{14cm}{Xc}
31 \hfill 1 & 0  \\
\hline
\multicolumn{1}{|c|}{Reserved} & 
\multicolumn{1}{|c|}{PWMxEN}  \\
\hline
\end{tabularx}
\caption{TMRxPWMCTL register}
\end{center}
\end{table}

\begin{description}

\item{0 - PWMxEN} \hfill \\
PWM enable
\end{description}





\subsection{Timer prescaler}
When enabled, the timer prescaler divides the main clock according to table \ref{timerprescaler}.

\begin{table}[H]
\begin{center}
\begin{tabularx}{6cm}{|c|c|}
\hline
Prescaler value & Clock divider \\
\hline
000 & 1 \\
\hline
001 & 2 \\
\hline
010 & 4 \\
\hline
011 & 8 \\
\hline  
100 & 16  \\
\hline
101 & 64  \\
\hline
110 & 256 \\
\hline
111 & 1024 \\
\hline
\end{tabularx}
\caption{Timer prescaler values}\label{timerprescaler}
\end{center}
\end{table}

\subsection{Timer PWM update policy}
Sometimes is desirable to use a double-buffering technique when manipulating PWM parameters, to avoid 
them changing at unwanted time.

\begin{table}[H]
\begin{center}    
\begin{tabular}{|c|c|}
\hline
UPDP value & Update Policy \\
\hline
00 & UPDATE NOW \\
\hline
01 & UPDATE ON ZERO SYNC \\
\hline
10 & UPDATE LATER \\
\hline
11 & Reserved \\
\hline
\end{tabular}
\caption{PWM update policy}\label{updatepolicy}
\end{center}
\end{table}

\begin{description}
\item{UPDATE NOW} \hfill \\
Update PWM values as you write them.
\item{UPDATE ON ZERO SYNC} \hfill \\
Update PWM values when the timer counter reaches 0. Until then the new PWM values will stay on the double buffer.
\item{UPDATE LATER} \hfill \\
Don't update PWM values, just update the double buffer.
\end{description}

For precise PWM synchronization, do:
\begin{itemize}
\item Set update policy to UPDATE LATER. 
\item Change the PWM parameters at will.
\item Set update policy to UPDATE ON ZERO SYNC. When timer counter reaches 0 (or overflows), all PWM will be set synchronously.
\end{itemize}

\subsection{Software}\label{timersoftware}
As of ZPUino 1.0, no software classes are implemented to manipulate the timer, meaning you have to write directly to its registers. \\
\subsubsection{Setting up a timer interrupt}

The following example shows how to use interrupts with Timer 0. 
Timer is set up for a 100KHz operation with interrupts enabled.
\begin{lstlisting}[language=C++]
void _zpu_interrupt()
{
  if (TMR0CRL & _BV(TCTLIF)) { /* Interrupt comes from timer 0 */
                
    /* do something here... */
                
    /* Clear the interrupt flag on timer register */
    TMR0CTL &= ~_BV(TCTLIF);
  }
}

void setup()
{
  unsigned frequency = 100000;
        
  TMR0CNT = 0;  /* Clear timer counter */
  TMR0CMP = (CLK_FREQ / frequency) - 1; /* Set up timer, no prescaler */
  TMR0CTL = _BV(TCTLENA)|_BV(TCTLCCM)|_BV(TCTLDIR)|_BV(TCTLIEN);

  INTRMASK = _BV(INTRLINE_TIMER0); /* Enable timer 0 interrupt on mask */
  INTRCTL = _BV(INTREN); /* Globally enable interrupts */
}
\end{lstlisting}




\subsubsection{PWM}
A few macros are provided to ease PWM register access: \\

\begin{description}
\item{TMR0PWMLOW(x)} \hfill \\ PWMLOW register on timer 0, for PWM output x
\item{TMR0PWMHIGH(x)}\hfill \\ PWMHIGH register on timer 0, for PWM output x
\item{TMR0PWMCTL(x)} \hfill \\ PWMCTL register on timer 0, for PWM output x

\item{TMR1PWMLOW(x)} \hfill \\ PWMLOW register on timer 1, for PWM output x
\item{TMR1PWMHIGH(x)}\hfill \\ PWMHIGH register on timer 1, for PWM output x
\item{TMR1PWMCTL(x)} \hfill \\ PWMCTL register on timer 1, for PWM output x

\end{description}
